%% start of file `resume_en.tex'.

\documentclass[11pt,a4paper]{moderncv}

% moderncv themes
\moderncvtheme[grey]{classic}                 
% optional argument are 'blue' (default), 'orange', 'red', 'green', 'grey' and 'roman' (for roman fonts, instead of sans serif fonts)

%\moderncvtheme[green]{classic}                % idem

% character encoding
\usepackage[utf8]{inputenc}                   % replace by the encoding you are using

% adjust the page margins
\usepackage[scale=0.8]{geometry}
\recomputelengths                             % required when changes are made to page layout lengths

% personal data
\firstname{Rafael}
\familyname{Gonçalves}
\title{Desenvolvedor e Analista}                    % optional, remove the line if not wanted
\mobile{+55(11)96729-1897}                    % optional, remove the line if not wanted
\email{rafaelolg@gmail.com}                      % optional, remove the line if not wanted
\extrainfo{\href{http://br.linkedin.com/in/rafaelolg}{http://br.linkedin.com/in/rafaelolg}}
 % optional, remove the line if not wanted

%\nopagenumbers{}                             % uncomment to suppress automatic page numbering for CVs longer than one page


%----------------------------------------------------------------------------------
%            content
%----------------------------------------------------------------------------------
\begin{document}
\maketitle
%\section{Objective}
%\cvline{}{My objective}

\cventry{2010 -- 2013}{Mestrado}{Universidade São Paulo}{}{}{ 
Área de Pesquisa: Visão Computacional e Machine Learning.
\newline{Orientador:}  Prof. Dr. Roberto Hirata.
}

\cventry{2004--2009}{Bacharel em Ciência da Computação}{ Universidade São Paulo }{}{Instituto de Matemática e Estatística}{}

\section{Experiência Profissional}
\cventry{2012 -- Presente}{Desenvolvedor}{\href{http://www.artis.com.br/eximius/index\_eximius.php}{Artis}}{São Paulo}{}{
Desenvolvimento de neuronavegador para cirurgia assistida por imagem médica.
\newline{} Aplicação de visão computacional e gráfica em Python utilizando VTK;
\newline{} Líder técnico de um pequeno time.
}

\cventry{2011 -- 2012}{Analista de Projetos}{\href{http://www.nic.br}{Nic.br}}{São Paulo}{}{
    Desenvolvimento e análise de tecnologias para medição da banda larga brasileira. 
\newline{} Análise de dados de resultados dos testes lidando com grandes datasets.
\newline{} Criação de mecanismo de detecção automática de falhas em rede usando machine learning.
\newline{} Desenvolvimento de webservice para storage distribuído.
\newline{} Trabalhando com C(embedded), Java e Python(para análise de dados).
\newline{} Projeto: \href{http://simet.nic.br}{Simet.nic.br}.
}
%
\cventry{2010 -- 2011}{Desenvolvedor}{\href{http://www.whitehat.com.br}{WhiteHat}}{São Paulo}{}{Desenvolvimento de aplicações Django.
\newline{} Criação de  webscrapper para extrair dados de redes sociais.
\newline{} Outsourcing para o IBOPE.
}

\cventry{2007 -- 2009}{Estágio em Desenvolvimento Software}{\href{http://www.pra.com.br}{PRA}}{São Paulo}{}{Desenvolvimento de software financeiro.
\newline{} Tecnologias utilizadas incluem Java Server Faces, Spring, Maven e JPA.}

%
\cventry{2007 -- 2007}{Estágio em Desenvolvimento Software}{\href{http://www.lsitec.org.br}{LSITEC}}{São Paulo}{}{Desenvolvimento de software embarcado em C.
\newline{} Desenvolvimento de celulares.
\newline{} Interação internacional.}


\cventry{2005 -- 2007}{Administrador de rede}{\href{http://www.linux.ime.usp.br}{ Rede Linux-IME-USP}}{São Paulo}{}{ Manutenção de uma rede baseada em Linux com mais de 100 computadores e 3000 mil usuários.}


\section{Habilidades em Computação}
\cvline{Linguagens}{Python, C, Java, SQL,  e boas noções de R e C++.}
\cvline{Markup}{HTML, CSS e \LaTeX.}
\cvline{Sistemas de Controle de Versão}{Subversion, GIT e Mercurial.}
\cvline{Platformas}{Linux, Unix.}
\cvline{Ferramentas}{GCC, Make, Autotools, Maven.}
\cvline{Frameworks Python}{Django, OpenCV, VTK, Numpy, Pandas, Scikit.}
\cvline{Frameworks J2EE}{Spring, Hibernate e EJB.}
\cvline{C Libraries}{Pthreads, sockets, OpenGL, OpenCV.}

\section{Línguas}
\cvline{Inglês}{Ler, escrever e conversação}

\section{Áreas de interesse em pesquisa}
\cvlistitem{Visão Computacional.}
\cvlistitem{Machine learning.}
\cvlistitem{Análise de Dados.}

\section{Interesses em Geral}
\cvline{Colaborações em projetos Open Source}{
\href{bitbucket.org/rafaellg/visiondataset}{bitbucket.org/rafaellg/visiondataset} --- Plataforma web para colaboração científica.
\newline{} \href{visocor.sourceforge.net}{visocor.sourceforge.net} --- Um plugin Opengl para o Compiz para prover acessibilidade a daltônicos. 
}

\closesection{}                   % needed to renewcommands
\renewcommand{\listitemsymbol}{-} % change the symbol for lists

%\section{Extra 1}
%\cvlistitem{Item 1}
%\cvlistitem{Item 2}
%\cvlistitem[+]{Item 3}            % optional other symbol

%\section{Extra 2}
%\cvlistdoubleitem[\Neutral]{Item 1}{Item 4}
%\cvlistdoubleitem[\Neutral]{Item 2}{Item 5}
%\cvlistdoubleitem[\Neutral]{Item 3}{}

% Publications from a BibTeX file
%\nocite{*}
%\bibliographystyle{plain}
%\bibliography{publications}       % 'publications' is the name of a BibTeX file

\end{document}

